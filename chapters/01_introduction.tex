\chapter{Inleiding}
\section{Labo-automatisatie}
In de afgelopen jaren hebben robotische processen de manier waarop we experimenten en de bijhorende labotaken uitvoeren veranderd. Geautomatiseerde systemen hebben, vooral in biomedisch en moleculair onderzoek, een enorm potentieel aangetoond voor het verhogen van de reproduceerbaarheid van experimenten, het stroomlijnen van experimentele methoden en het verminderen van de impact van menselijke fouten.\cite{RN5} Deze thesis zal zich richten op de ontwikkeling van een robotisch pipetteersysteem dat is afgestemd op de behoeften van de onderzoeksgroep voor Translational Neurosciences aan de Universiteit Antwerpen. Zij trachten hiermee de robuustheid en reproduceerbaarheid van hun experimenten te verbeteren. Dit gebeurt in het kader van een bredere automatisering van hun Labo. Door een van de kernuitdagingen in experimentele reproduceerbaarheid aan te pakken probeert dit werk bij te dragen aan bredere inspanningen om betrouwbare onderzoeksinstrumenten te creëren.

\section{Breder kader}
\subsection{Reproduceerbaarheidscrisis}
Reproduceerbaarheid is een hoeksteen van wetenschappelijk onderzoek en zorgt ervoor dat resultaten onafhankelijk kunnen worden geverifieerd en uitgebreid. Studies hebben echter gewezen op een groeiende reproduceerbaarheidscrisis in onder meer biomedisch onderzoek, veroorzaakt door inconsistenties in handmatige procedures, subjectieve beoordelingen en omgevingsvariabiliteit.\ \cite{RN2} De reproduceerbaarheidscrisis verwijst naar de moeilijkheid om wetenschappelijke resultaten consistent te repliceren, een probleem dat zowel technische als sociale oorzaken kent. Dit gebrek aan redproductie heeft een significante impact op bijvoorbeeld kankeronderzoek en andere gerelateerde onderzoeks-disciplines.\ \cite{RN4}
\subsection{Robotic labs}
Onderzoekers kijken onder andere naar robotische labo’s als oplossing. Dit zijn labo’s waarbij een deel van, of alle, taken worden uitgevoerd door robots. Deze dragen als voordeel met zich mee dat repetitieve taken niet langer door mensen moeten uitgevoerd worden. Deze kunnen dan aan lage kost parallel uitgevoerd worden. Dit heeft in bijvoorbeeld genoomonderzoek al voor een grote versnelling gezorgd doorheen de laatste twee decennia. Dit komt doordat labo’s nu sneller stalen kunnen analyseren. Onderzoek toont aan dat deze versnelling grotendeels door geautomatiseerde labo’s word gedreven en dat niet geautomatiseerde labo’s zelfs achterlopen.[4]

\subsection{Bron van de reproduceerbaarheidscrisis}
De reproduceerbaarheidscrisis heeft diverse oorzaken. Vaak word er gekeken naar de sociale en economische aspecten van wetenschap.\cite{RN12} Hierbij is de druk om origineel onderzoek te publiceren zo groot dat dit een druk uitoefent op de bestaande systemen van peer-evaluatie en worden resultaten soms herwerkt tot ze een significante ontdekking vertonen met methodes als “p-hacking”.\cite{RN2,RN6} Er zijn echter ook technische beperkingen. Zo worden, door menselijke fouten, handelingen niet altijd uitgevoerd zoals ze in het experiment beschreven staan. 

\section{Robotic labs als oplossing voor reproduceerbaarheidscrisis}
Door het automatiseren van enkele of alle taken wordt de mogelijkheid tot menselijke fout verlaagd. Robots kunnen ervoor zorgen dat repetitieve taken telkens op dezelfde manier worden uitgevoerd. Bij het manueel pipeteren zijn dit vaak taken waar vermoeidheid en fysieke klachten voor variatie kunnen zorgen.\ \cite{RN9} Doordat deze systemen programmeerbaar zijn en deterministisch werken is het mogelijk om de exacte handelingen te delen zonder de kans dat stappen worden weggelaten.
\section{Pipetrobot als onderdeel van Robotic lab}
Geautomatiseerde labo’s hebben echter nog een belangrijk minpunt. De investeringskosten zijn groot en het automatiseren van een volledig labo kan lang duren. De efficiëntie-winst wordt steeds beperkter naarmate de doorvoer van het labo de werkbelasting bereikt. Voor veel, kleinere, labo’s is dit dan ook niet altijd interessant. Als compromis stellen onderzoekers semi-automatisatie voor. Hierbij werd gekeken naar welke stappen de grootste impact hebben en concludeerde men dat de belangrijkste winsten voortkomen uit het automatiseren van de meest repetitieve en tijdrovende taken. Hierbij werd hoofdzakelijk pipetteren als kandidaat gezien.\cite{RN11,RN7}
\section{Probleemstelling}
Als einddoel zal er getracht worden om een systeem te voorzien dat met een API zal toelaten om programmatisch pipet-handelingen te ondernemen. Door deze zo kosten-efficiënt mogelijk te ontwerpen is het de bedoeling dat deze robot toegankelijk zal zijn voor labo’s die de hoge investeringskosten van deze automatiseringssystemen willen vermijden. De thesis zal zich specifiek richten op het beantwoorden van de vraag:
\begin{quote}
    \begin{center}
        \textbf{\textit{Welke ontwerp- en implementatievereisten zijn nodig voor de ontwikkeling van een programmeerbare, kosten-efficiënte pipetrobot die reproduceerbaarheid in laboratoria verbetert?}}
    \end{center}
\end{quote}
We zullen ons hiervoor baseren op bestaande oplossingen van gelijkaardige problemen. Hierbij is het ontwerp van een 3-assige “cartesian gantry robot”, zoals bijvoorbeeld de “Opentrons OT2” en veel 3D-printers, een mogelijke beginpiste. Dit biedt flexibiliteit en een eenvoudige uitvoering. De interactie met de motoren zal grotendeels via C++ plaatsvinden voor lage-latentie besturing. Python zal de programmeertaal zijn waar de eindgebruiker pipetteertaken via een API kan aansturen. Hierbij is gebruiksgemak namelijk belangrijk en is python, door zijn toegankelijkheid een geschikte kandidaat.
Een van de belangrijkste doelstellingen is het creëren van een kostenefficiënte oplossing voor laboratoria die zich geen dure, volledig geautomatiseerde systemen kunnen veroorloven. Door een aangepast ontwerp aan te bieden, kunnen laboratoria kiezen voor een oplossing die voldoet aan hun specifieke behoeften zonder overbodige kosten. Dit betekent ook dat er open source software en hardware gebruikt kan worden om de initiële kosten te drukken, en tegelijkertijd flexibiliteit en uitbreidbaarheid te bieden. Dit biedt een aantrekkelijk alternatief voor de dure commerciële systemen die vaak voorgeconfigureerd zijn en weinig ruimte laten voor aanpassing. De mogelijk om het systeem via een GUI te bestuderen wordt meegenomen in de einddoelstellingen. Dit zal echter afhangen van de vooruitgang van de andere, meer prioritaire, aspecten.