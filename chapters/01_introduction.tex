\chapter{Inleiding}
\section{laboratorium-automatisatie}
In de afgelopen jaren hebben robotische processen de manier waarop we experimenten en de bijhorende labotaken uitvoeren veranderd. Geautomatiseerde systemen hebben, vooral in biomedisch en moleculair onderzoek, een enorm potentieel aangetoond voor het verhogen van de reproduceerbaarheid van experimenten, het stroomlijnen van experimentele methoden en het verminderen van de impact van menselijke fouten.\cite{RN5} 
\\[12pt]Deze thesis zal zich richten op de ontwikkeling van een robotisch pipetteersysteem dat is afgestemd op de behoeften van de onderzoeksgroep \'Translational Neurosciences' aan de Universiteit Antwerpen. Het doel van dit onderzoek is om de robuustheid en reproduceerbaarheid van hun experimenten te verbeteren. Dit gebeurt in het kader van een bredere automatisering van hun laboratorium. Door een van de kernuitdagingen in experimentele reproduceerbaarheid, namelijk het maken van technische fouten bij repitief werk, aan te pakken probeert dit werk bij te dragen aan bredere inspanningen om betrouwbare onderzoeksinstrumenten te creëren.

\section{Breder kader}

\subsection{Reproduceerbaarheidscrisis}
Reproduceerbaarheid is een hoeksteen van wetenschappelijk onderzoek en zorgt ervoor dat resultaten onafhankelijk kunnen worden geverifieerd. Studies hebben echter gewezen op een groeiende reproduceerbaarheidscrisis in onder meer biomedisch onderzoek, veroorzaakt door inconsistenties in handmatige procedures, subjectieve beoordelingen en omgevingsvariabiliteit.\ \cite{RN2} De reproduceerbaarheidscrisis verwijst naar de moeilijkheid om wetenschappelijke resultaten consistent te repliceren, een probleem dat zowel technische als menselijke oorzaken kent. 
\\[12pt]Vaak wordt er gekeken naar de sociale en economische aspecten van wetenschap.\cite{RN12} De druk om origineel onderzoek te publiceren is zo groot dat dit een druk uitoefent op de bestaande systemen van peer-evaluatie en soms worden resultaten herwerkt tot ze een significante ontdekking vertonen met methodes als “p-hacking”.\cite{RN2,RN6} Er zijn echter ook technische beperkingen. Zo worden door menselijke fouten handelingen niet altijd uitgevoerd zoals ze in het experiment beschreven staan. 
\\[12pt]Dit gebrek aan succesvolle reproductie heeft een significante impact op bijvoorbeeld kankeronderzoek en andere gerelateerde onderzoeks disciplines.\ \cite{RN4}
\subsection{Robotic labs}
Onderzoekers kijken onder andere naar robotische laboratoria als oplossing. Dit zijn laboratoria waarbij een deel van, of alle, taken worden uitgevoerd door robots. Deze dragen als voordeel met zich mee dat repetitieve taken niet langer door mensen moeten uitgevoerd worden. Deze kunnen dan aan lage kost parallel uitgevoerd worden. Dit heeft in bijvoorbeeld genoomonderzoek al voor een grote versnelling gezorgd doorheen de laatste twee decennia. Dit komt doordat laboratoria sneller hun stalen kunnen analyseren. Onderzoek toont aan dat deze versnelling grotendeels door geautomatiseerde laboratoria wordt gedreven en dat niet geautomatiseerde laboratoria zelfs achterlopen.[4]

\section{Robotic labs als oplossing voor reproduceerbaarheidscrisis}
Door het automatiseren van enkele of alle taken wordt de mogelijkheid tot menselijke fout verminderd. Robots kunnen ervoor zorgen dat repetitieve taken telkens op dezelfde manier worden uitgevoerd. Bij het manueel pipetteren kunnen deze taken door vermoeidheid en fysieke klachten voor variatie kunnen zorgen.\ \cite{RN9} Doordat liquid handling robots programmeerbaar zijn en deterministisch werken is het mogelijk om de exacte handelingen te delen zonder het risico dat stappen worden weggelaten.

\section{Pipetrobot als onderdeel van Robotic lab}
Geautomatiseerde laboratoria hebben echter nog een belangrijk minpunt. De investeringskosten zijn groot en het automatiseren van een volledig laboratorium kan lang duren. 
Voor veel kleinere laboratoria is volledig automatisatie dan ook niet altijd interessant. Als compromis stellen onderzoekers semi-automatisatie voor. Hierbij word gekeken naar welke stappen de grootste impact hebben en concludeert men dat de belangrijkste winsten voortkomen uit het automatiseren van de meest repetitieve en tijdrovende taken. Hierbij werd hoofdzakelijk pipetteren als kandidaat gezien.\cite{RN11,RN7}

\section{Probleemstelling}
Als einddoel zal er getracht worden om een systeem te ontwikkelen dat (met een API) zal toelaten om programmatisch pipet-handelingen uit te voeren. Het systeem zal zo ontworpen worden dat het geïntegreerd kan worden met bestaande robots, als end effector. Door deze zo kosten-efficiënt mogelijk te ontwerpen is het de bedoeling dat deze robot toegankelijk zal zijn voor laboratoria die de hoge investeringskosten van deze automatiseringssystemen willen vermijden. De thesis zal zich specifiek richten op het beantwoorden van volgende onderzoeksvraag:\begin{quote}
    \begin{center}
        \textbf{\textit{Welke ontwerp- en implementatievereisten zijn nodig voor de ontwikkeling van een programmeerbare, kosten-efficiënte pipetrobot die reproduceerbaarheid in laboratoria verbetert?}}
    \end{center}
\end{quote}
Door een aanpasbaar en modulair ontwerp aan te bieden, kunnen laboratoria kiezen voor een oplossing die voldoet aan hun specifieke behoeften zonder overbodige kosten. Dit betekent ook dat er reeds bestaand software en hardware gebruikt kan worden om enerzijds de initiële kosten te drukken, en eveneens flexibiliteit en uitbreidbaarheid te bieden. Dit biedt een aantrekkelijk alternatief voor de dure commerciële systemen die vaak voorgeconfigureerd zijn en weinig ruimte laten voor aanpassing.