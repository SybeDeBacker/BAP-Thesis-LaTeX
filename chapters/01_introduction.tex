\chapter{Inleiding}

In de afgelopen jaren hebben robotische processen de manier waarop we experimenten en de bijhorende labotaken uitvoeren, sterk veranderd. Geautomatiseerde systemen hebben, vooral in het biomedisch en moleculair onderzoek, een aanzienlijk potentieel aangetoond om de reproduceerbaarheid van experimenten te verhogen, experimentele methoden te stroomlijnen en de impact van menselijke fouten te verminderen\ \cite{RN5}.
\\[12pt]Deze thesis richt zich op de ontwikkeling van een robotisch pipetteersysteem dat is afgestemd op de behoeften van de onderzoeksgroep \textit{Translational Neurosciences} aan de Universiteit Antwerpen. Het doel is om de robuustheid en reproduceerbaarheid van hun experimenten te verbeteren in het kader van een bredere automatisering van het laboratorium. Door één van de kernuitdagingen in experimentele reproduceerbaarheid aan te pakken — namelijk menselijke fouten bij repetitieve handelingen — poogt dit werk bij te dragen aan het creëren van betrouwbare onderzoeksinstrumenten.
\\[12pt]Reproduceerbaarheid is een fundament van wetenschappelijk onderzoek, omdat het garandeert dat resultaten onafhankelijk kunnen worden geverifieerd. Toch wijzen diverse studies op een groeiende reproduceerbaarheidscrisis binnen onder andere het biomedisch onderzoek. Deze crisis wordt deels veroorzaakt door inconsistenties in handmatige procedures, subjectieve beoordelingen en variabiliteit in omgevingsomstandigheden\ \cite{RN2}. 
\\[12pt]Er wordt vaak gewezen op de sociale en economische druk binnen de academische wereld, zoals de sterke focus op het publiceren van originele resultaten. Dit ondermijnt bestaande evaluatieprocessen, en leidt soms tot praktijken zoals \textit{p-hacking}, waarbij data net zo lang worden geanalyseerd tot significante resultaten gevonden worden\ \cite{RN2,RN6}. Naast deze structurele factoren zijn er ook technische beperkingen: menselijke fouten zorgen ervoor dat handelingen niet altijd worden uitgevoerd zoals oorspronkelijk voorgeschreven.
\\[12pt]Zoals beschreven in het werk van Begley en Ellis\ \cite{RN4}, heeft dit gebrek aan reproduceerbaarheid grote implicaties voor onder andere het kankeronderzoek en aanverwante disciplines. Door laboratoriumtaken gedeeltelijk of volledig te automatiseren, kan het risico op menselijke fouten drastisch worden verlaagd. Robots voeren repetitieve taken op een consistente manier uit, terwijl menselijke operaties zoals manueel pipetteren vaak lijden onder vermoeidheid en fysieke belasting\ \cite{RN9}. 
\\[12pt]Omdat zogeheten \textit{liquid handling}-robots programmeerbaar en deterministisch zijn, kunnen zij experimentele handelingen exact reproduceren, zonder het risico dat stappen worden overgeslagen of verkeerd worden uitgevoerd.

\section{Probleemstelling}
Het uiteindelijke doel van dit onderzoek is de ontwikkeling van een systeem dat via een API programmeerbare pipetteerhandelingen kan uitvoeren. Dit systeem zal zodanig worden ontworpen dat het als \textit{end effector} op bestaande robots kan worden gemonteerd. Door te streven naar een kostenefficiënt ontwerp beoogt dit werk een oplossing te bieden die toegankelijk is voor laboratoria waarvoor de hoge investeringskosten van bestaande automatiseringssystemen een drempel vormen.
\\[12pt]De centrale onderzoeksvraag luidt als volgt:
\begin{quote}
    \begin{center}
        \textbf{\textit{Welke ontwerp- en implementatievereisten zijn nodig voor de ontwikkeling van een programmeerbare, kostenefficiënte pipetrobot die reproduceerbaarheid in laboratoria verbetert?}}
    \end{center}
\end{quote}

Door een modulair en aanpasbaar ontwerp aan te bieden, kunnen laboratoria kiezen voor een oplossing die voldoet aan hun specifieke behoeften, zonder overbodige kosten. Dit betekent ook dat bestaande software en hardware kunnen worden hergebruikt, wat niet alleen de initiële kost verlaagt, maar ook flexibiliteit en uitbreidbaarheid bevordert. Deze aanpak biedt een aantrekkelijk alternatief voor de vaak dure commerciële systemen, die doorgaans weinig ruimte laten voor maatwerk of integratie met andere componenten.
