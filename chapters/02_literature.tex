\chapter{Achtergrond}


\section{Pipetteconcepten}
\subsection{Air-Displacement}
%\subsubsection{Eerste pipetten}
%Mondpipetten waren doorheen de 19e en in het begin van de 20e eeuw de standaard voor het pipetteren. In \cite{RN21} wordt bijvoorbeeld een ontwerp voor een mondpipet voorgesteld om steriliteitstesten mee uit te voeren. In 1950 werd door A. J. Swallow in\cite{RN18} de eerste mechanische pipet voorgesteld voor het pipetteren van radioactieve stoffen. Hierdoor was er geen contact mogelijk tussen de mond en de te pipetteren vloeistof. Dit helpt ook bij het vermijden van besmetting van labopersoneel zoals beschreven in \cite{RN20}.
\subsubsection{Theoretische achtergrond}
In\ \cite{RN15} staat beschreven hoe een air-displacement pipet werkt via de ideale gaswet. Er wordt een omgeving van lagere druk gecreëerd door het veranderen van het volume van de pipet voor aspiratie. Dit volume wordt ingenomen door de vloeistof waar de pipet zich in bevindt. Bij een mondpipet wordt deze negatieve druk gecreëerd door de longen van de operator. Bij de mechanische pipet wordt deze via de peer gecreëerd door deze initieel in te drukken en daarna terug te laten opvullen.
\subsubsection{Analoge pipetten}
Analoge pipetten, zoals beschreven in o.a.\ de patenten\ \cite{RN16} en\ \cite{RN17}, werken op basis van een zuigerwerking. De slag van de zuiger kan voor de aspiratie bepaald worden door middel van een wiel in geval van\ \cite{RN17} of door twee elementen in elkaar te schroeven in geval van\ \cite{RN16}. Bij\ \cite{RN17} zal dit wiel, door het axiaal verschuiven van een loodschroef, de zuigerstang verschuiven. In rustpositie zal deze dus korter lijken bij een kleiner ingesteld volume. De zuigerstang kan dan ingedrukt worden tot een stop-nut, deze blijft altijd op de zelfde plaats. Doordat de beginpositie aangepast wordt, wordt ook de slag van de zuiger aangepast. Hiermee wordt het volume bepaald.\ \cite{RN16} volgt een gelijkaardig principe. Hier zal echter de eindpositie bepaald worden door het onderste deel verder te schroeven. De analoge pipetten beschreven in deze patenten gebruiken air-displacement. In geval van\ \cite{RN17} gebeurt dit met een zuiger (44) die relatief verder van de pipetteer-punt (28) staat dan bij\ \cite{RN16}.\ \cite{RN16} is eenvoudiger uitgevoerd dan\ \cite{RN17}. Er zijn minder dichtingen en er is geen display om het gewenste volume van af te lezen. Dit maakt echter ook dat\ \cite{RN16} meer problemen zal hebben op vlak van lekkage en dus precisie.

\subsection{Positive Displacement}