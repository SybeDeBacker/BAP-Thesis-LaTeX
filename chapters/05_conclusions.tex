\chapter{Conclusies}
In deze bachelorproef werd een modulair, kostenefficiënt pipetteersysteem ontworpen en gerealiseerd met het oog op het verbeteren van reproduceerbaarheid in laboratoria. De opzet was om een systeem te ontwikkelen dat als end effector kan functioneren binnen een groter geautomatiseerd laboratorium, met specifieke toepassing binnen de onderzoeksgroep \textit{Translational Neurosciences} aan de Universiteit Antwerpen.
\\[12pt]Door een combinatie van 3D-geprinte componenten, een nauwkeurig aangedreven lineaire beweging via een loodschroef, en een aanstuurbare microcontroller werd een robuust systeem gerealiseerd. De resultaten van de nauwkeurigheidstesten toonden aan dat de pipet voldoet aan de ISO 8655 normen (\cite{RN50}), wat de geschiktheid van dit ontwerp voor laboratoriumgebruik bevestigt.
\\[12pt]Samenvattend kan worden gesteld dat de ontwerp- en implementatievereisten succesvol zijn ingevuld. Dit pipetteersysteem biedt een waardevol alternatief voor dure commerciële oplossingen, met behoud van precisie, reproduceerbaarheid en gebruiksgemak.
\\[12pt]\textbf{Aanbevelingen voor toekomstig werk} zijn onder andere:
\begin{itemize}
  \item Integratie van closed-loop feedback voor verhoogde precisie.
  \item Ontwikkeling van een multi-channel versie voor hoger throughput.
  \item Validatie in een reële laboratoriumsetting met biologische monsters.
\end{itemize}