\chapter{Resultaten}
Voor de ingestelde volumes voldoet de opstelling aan de ISO 8655 toleranties van maximaal ± 4 $\mu L$ systematische fout en ± 1{,}5 $\mu L$ random fout. Bijvoorbeeld, voor een ingesteld volume van 200 $\mu L$ waren de gemiddelde volumes respectievelijk 203{,}9 $\mu L$, 201{,}5 $\mu L$, 201{,}8 $\mu L$ en 202{,}8 $\mu L$, met systematische fouten tussen 1{,}5 en 3{,}9 $\mu L$ en random fouten onder 1{,}3 $\mu L$, zie ook \autoref{tab:accuracies}.

\begin{table}[H] 
    \centering 
    \caption{Resultaten van nauwkeurigheidstesten van de opstelling (n=30)}
    \begin{tabular}{|c|c|c|c|}
        \hline
        \textbf{Gewenste Volume [$\mu L$]} & \textbf{Gemiddelde [$\mu L$]} & \textbf{Systematische fout [$\mu L$]} & \textbf{Willekeurige fout [$\mu L$]}\\
        \hline
        100 & 99.9 & –0.1 & 0.67 \\
        200 & 203.9 & 3.9 & 0.51 \\ 
        300 & 000 & 000 & 000 \\
        400 & 000 & 000 & 000 \\
        500 & 000 & 000 & 000 \\
        600 & 000 & 000 & 000 \\
        700 & 000 & 000 & 000 \\
        800 & 000 & 000 & 000 \\
        900 & 000 & 000 & 000 \\
        1000 & 000 & 000 & 000 \\   
        \hline
    \end{tabular}\label{tab:accuracies}
\end{table}

Deze resultaten tonen aan dat het systeem voldoet aan de eisen van ISO 8655 voor zowel precisie als nauwkeurigheid, en dus geschikt is voor gebruik in laboratoriumomgevingen met hoge kwaliteitseisen.
Hierbij wordt het voorbeeld gevolgd van\ \cite{RN49}. Hier wordt ook een spuit-gebaseerd ontwerp getest op nauwkeurigheid met een beschreven procedure.