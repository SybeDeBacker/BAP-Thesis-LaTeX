\chapter{Resultaten}
Voor de ingestelde volumes voldeden alle vier de kanalen aan de ISO 8655 toleranties van maximaal ±4$\mu L$ systematische fout en ±1{,}5$\mu L$ random fout. Bijvoorbeeld, voor een ingesteld volume van 200$\mu L$ waren de gemiddelde volumes respectievelijk 203{,}9$\mu L$, 201{,}5$\mu L$, 201{,}8$\mu L$ en 202{,}8$\mu L$, met systematische fouten tussen 1{,}5 en 3{,}9$\mu L$ en random fouten onder 1{,}3$\mu L$, zie ook Tabel\ref{tab:accuracies}.

\begin{table}[H] \centering \begin{tabular}{c|c|c|c} 
    \textbf{Gewenste Volume} & \textbf{Gemiddelde ($\mu L$)} & \textbf{Systematische fout ($\mu L$)} & \textbf{Random fout ($\mu L$)}\\
    \hline
    100 & 99.9 & –0.1 & 0.67 \\
    200 & 203.9 & 3.9 & 0.51 \\ 
    300 & 000 & 000 & 000 \\
    400 & 000 & 000 & 000 \\
    500 & 000 & 000 & 000 \\
    600 & 000 & 000 & 000 \\
    700 & 000 & 000 & 000 \\
    800 & 000 & 000 & 000 \\
    900 & 000 & 000 & 000 \\
    1000 & 000 & 000 & 000 \\   

\end{tabular} \caption{Resultaten van nauwkeurigheidstesten per kanaal (n=10).}\label{tab:accuracies} \end{table}

Deze resultaten tonen aan dat het systeem voldoet aan de eisen van ISO 8655 voor zowel precisie als nauwkeurigheid, en dus geschikt is voor gebruik in laboratoriumomgevingen met hoge kwaliteitseisen
Hierbij wordt het voorbeeld gevolgd van\ \cite{RN49}. Hier wordt ook een spuit-gebaseerd ontwerp getest op nauwkeurigheid met een beschreven procedure.