\chapter{Resultaten}
Voor de ingestelde volumes voldoet de opstelling aan de ISO 8655--2\ \cite{RN51} toleranties, zie \autoref{tab:accuracies} kolom 4 en 6. Er moet wel rekening gehouden met een constante afwijking.\footnote{Als er wordt gevraagd om een volume $V_{wens}$ te pipetteren maar de pipet pipetteert $V_{wens}-V_{fout}$ dan wordt er een correctiefacor = $V_{wens}=V_{fout}$ toegepast zodat $V_{pipet} = V_{wens}+V_{correctie}-V_{fout} = V_{wens}$} Deze verschilde tussen de twee spuiten. Eens deze zijn toegepast (in de Arduino-code) bekomen we volgende resultaten:
\begin{table}[H] 
    \centering 
    \caption{Resultaten van nauwkeurigheidstesten van de opstelling \textit{(n=10)}.
    \\Voor alle metingen: zie \autoref{tab:resultaten_spuit_1} en \autoref{tab:resultaten_spuit_2}.}
    \begin{tabular}{|c|c|c|c|c|c|}
        \hline
        \textbf{\rule{0pt}{3ex}$V_{wens}$ [$\mu L$]} & 
        \textbf{$\bar{V}$ [$\mu L$]} & 
        \textbf{$e_s$ [\%]} & 
        \textbf{ISO max $e_s$ [\%]} &
        \textbf{$S_r$ [\%]} &
        \textbf{ISO max $S_r$ [\%]} \\
        \hline
        \multicolumn{6}{|c|}{\textbf{Spuit 1}} \\
        \multicolumn{6}{|c|}{\textit{correctiefactor constante fout: + $8,5\mu L$}}\\
        \hline
        100  & 100,2 & 0,21 & 8,0  & 0,63 & 3,0 \\
        500  & 502,6 & 0,52  & 1,6  & 0,20 & 0,6 \\
        1000 & 1003,1  & 0,31  & 0,8  & 0,12 & 0,3 \\
        \hline
        \multicolumn{6}{|c|}{\textbf{Spuit 2}}\\
        \multicolumn{6}{|c|}{\textit{correctiefactor constante fout: + $13,5\mu L$}}\\
        \hline
        100  & 100,8 & 0,78 & 8,0  & 0,87 & 3,0 \\
        500  & 500,9 & 0,18  & 1,6  & 0,13 & 0,6 \\
        1000 & 1001,9  & 0,19  & 0,8  & 0,16 & 0,3 \\
        \hline
    \end{tabular}\label{tab:accuracies}
\end{table}


Deze resultaten tonen aan dat het systeem voldoet aan de eisen van ISO 8655--2 voor zowel precisie als nauwkeurigheid, en dus geschikt is voor gebruik in laboratoriumomgevingen met hoge kwaliteitseisen.